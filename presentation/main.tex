%! TEX program = lualatex
%! BIB program = biber
\documentclass[aspectratio=1610, xcolor={dvipsnames}]{beamer}

\usetheme[
    % options
]{metropolis}

\setbeamercolor{background canvas}{bg = white}

\input{macro}
\usepackage{emoji}
\usepackage[normalem]{ulem}

\title{Trampolining: Dark Arts Edition}

\begin{document}

\begin{frame}
    % hello
    \centering
    \hrule \vspace{1em}
    {\Huge Trampolining \\ \vspace{0.2em}\(\mathfrak{Dark~ Arts~ Edition ~Part ~2}\)} \\
    \vspace{1.5em} \hrule 
    {\Huge \emoji{crystal-ball} \emoji{magic-wand}}
\end{frame}

\begin{frame}
    % hello
    \centering
    \hrule \vspace{1em}
    {\Huge Combinatory Logic, and an aside to GHC} \\
    \vspace{1.5em} \hrule 
    {\Huge \emoji{prohibited} \emoji{crystal-ball} \emoji{magic-wand} \emoji{prohibited}}
\end{frame}

\begin{frame}[fragile]
    \frametitle{Motivation}
\centering

\begin{tikzpicture}
    \node (code) at (0, 0) { 
        \begin{lstlisting}[language=haskell,numbers=none]
        double :: [a] -> [a]
        double [] = []
        double (hd : tl) = hd : hd : double tl
        \end{lstlisting}
     };

     \node (von) at (-3, -2) {Von Neumann Machine};
     \node (fol) at (5, -2) {First-order logic};

     \draw[->] (code) -- (von);
     \draw[->] (code) -- (fol);

     \onslide<2-> {
        \node (fun) at (1, -4) {Doesn't know what a ``function'' is};
        \draw[->] (fun) -- (von);
        \draw[->] (fun) -- (fol);
     }
\end{tikzpicture}

% so from what we've been seeing of trampolining, the solution is to trade
% computation for data.

\end{frame}

\begin{frame}[fragile]
    \frametitle{Recap}

    Task: Take a list of size $1e7$, double it, and calculate its length.

    \begin{lstlisting}[language=bash]
    $ scala-cli run scala-list.sc
    // DNF: stack overflow
    \end{lstlisting}

    \begin{lstlisting}[language=bash]
    $ scala-cli run scala-defunc.sc
    // 1240 ms 
    \end{lstlisting}

    \begin{lstlisting}[language=bash]
    $ sbtn nativeLink && ./target/scala-3.3.3/scaladefunc
    // defunctionalized version, no code changes
    // 6046 ms (237 ms for 1e6)
    \end{lstlisting}

    \begin{lstlisting}[language=bash]
    $ stack ghc --resolver nightly haskell-list.hs && ./haskell-list
    // 67 ms
    \end{lstlisting}

\end{frame}

\begin{frame}
    \frametitle{Sources}

    Discussions with Shardul, Simon, and Viktor.

    Several papers, added at the end.

\end{frame}

\begin{frame}
    \frametitle{First-order logic, optional recap}

    A first-order structure:

    \begin{itemize}
        \item A domain / universe \(U\)
        \item A set of relations \(R\)
        \item A set of functions \(F\)
    \end{itemize}

    From the logic:
    \begin{itemize}
        \item A set of variables \(V\)
        \item Logical connectives \(\land, \neg\)
        \item Binder \(\forall\)
    \end{itemize}

    Notably, the relations and functions are \emph{not} first-class objects,
    only elements of the universe are. We may write \(f(a)\) but never simply
    \(f\).

\end{frame}

\begin{frame}
    \frametitle{Functions as data: slightly more formal}

    \begin{equation*}
        \lambda x.\; \lambda y.\; x \;(y \;z)
    \end{equation*}

    \begin{itemize}
        \item Free variable, \(z\)
        \item Bound variables \(x, y\)
    \end{itemize}

    Can we simply write \(f = \lambda x.\; \lambda y.\; x \;(y \;z) \implies f \;a =  \lambda y.\; a\; (y\; z)\) ?

    \pause

    Could we write the LHS as \(\forall x.\; \forall y.\; f(x, y) = x(y(z))\)?

\end{frame}

\begin{frame}
    \frametitle{Combinatory Logic}

    \begin{quote}
        ``Combinatory logic is a notation to eliminate the need for quantified variables in mathematical logic.''
    \end{quote}

    \pause

    \begin{definition}[Baren]
        A combinator is a lambda expression which contains no occurrences of a free variable.
    \end{definition}

    Standard example, the \(S\), \(K\), and \(I\) combinators:
    \begin{align*}
        S \;x \;y \;z &= x \;z \;(y \;z) \tag{Verschmelzungsfunktion}\\
        K \;x \;y &= x \tag{Konstanzfunktion}\\
        I \;x &= x \tag{Identit\"atsfunktion}
    \end{align*} 

    \pause
    Older than the lambda calculus, Sch\"onfinkel (1924) and Curry (1930).

\end{frame}

\begin{frame}
    \frametitle{`Compiling' to combinatory logic}

    Starting from the innermost abstraction:
    \begin{align*}
        \lambda x.\; x & \mapsto I \\
        \lambda x.\; c & \mapsto K \;c \qquad (x \not \in fv(c), bv(c) = \emptyset) \\
        \lambda x.\; e_1\; e_2 & \mapsto S \; (\lambda x.\; e_1) \;(\lambda x.\; e_2)
    \end{align*}
    \pause
    \begin{align*}
        &&\lambda x.\; \lambda y.\; x \;(y \;z) \\
        &\onslide<2->{\mapsto} &  \onslide<3->{\lambda x.\; S \;(\lambda y. x) \;(\lambda y. y \;z)}\\
        &\onslide<3->{\mapsto} &  \onslide<4->{\lambda x.\; S \;(K \;x) \;(\lambda y. y \;z)}\\
        &\onslide<4->{\mapsto} &  \onslide<5->{\lambda x.\; S \;(K \;x) \;(S \; (\lambda y. \; y) (\lambda y. \; z))}\\
        &\onslide<5->{\mapsto} &  \onslide<6->{\lambda x.\; (S \;(K \;x)) \;(S \; (I \;(K \;z)))}\\
        &\onslide<6->{\mapsto} &  \onslide<7->{S\; (\lambda x. \;S \;(K \;x)) \;(\lambda x.\; S \; (I \;(K \;z)))}\\
        &\onslide<7->{\mapsto} &  \onslide<8->{S\; ((K\; S) \;((K\; K) \;I)) \;(K \; (S \; (I \;(K \;z))))}\\
    \end{align*}
\end{frame}

\begin{frame}
    \frametitle{Takeaways}

    % I don't mean to say any of these are pros and cons of this system, just
    % what we can learn from this

    \begin{itemize}[<+->]
        \item Eliminated bound variables (yay!)
        \item Evaluation rules are like rewrites
        \item In particular, cannot partially evaluate
        \item But the generated term is huge (possibly cubic)
        \item The term we generated was not minimal, but minimal terms are not
        unique and hard to find
    \end{itemize}

    % I think xyz are pros, xyz are cons
    % we don't know hwo to address the last one, but we can try to do something about this one
    % any suggestions?

    % well the reason these terms are large is ebcause our abstractions are somewhat minimal and strict
    % in that sense we see that this highlights a really cool thing about the LC, that it's incredibly tiny but very expressive wrt term size

    % so, we can try to expand our abstractions a bit, and for a bit, jump to an extreme, which is supercombinator compilation

\end{frame}

\begin{frame}
    \frametitle{Supercombinator compilation}

    % otherwise also called lambda-lifting
    Motivation\footnote{Simon Peyton Jones (1987). The Implementation of Functional Programming Languages.}
    \begin{equation*}
        (\lambda x.\; \lambda y.\; - \;y \; x) \;3 \;4
    \end{equation*}
    %
    \begin{itemize}[<+->]
        \item We gain much: no intermediate term, one less step
        \item We lose nothing wrt computation: the intermediate term is useless
        % I quote, hence we may as well wait until both arguments are present and then perform both reductions at once
        \item Issue: our computation model now needs to accommodate arbitrary arity reductions
    \end{itemize}

\end{frame}

\begin{frame}
    \frametitle{Supercombinators}

    Solution: arbitrary arity reductions \emph{are} the computation model.

    \begin{definition}
        A supercombinator \(\$S\) of arity \(n\) is a lambda expression of the form:
        \begin{equation*}
            \lambda x_1.\; \lambda x_2.\; \ldots \lambda x_n.\; E
        \end{equation*}
        
        where
        \begin{itemize}
            \item \(E\) is not a lambda abstraction
            \item any lambda abstraction in \(E\) is a supercombinator itself
            \item \(n \geq 0\)
        \end{itemize}
    \end{definition}

    \pause
    Each supercombinator is paired with a \emph{reduction} acting on a redex. A
    \emph{supercombinator redex} is a fully applied supercombinator. 

\end{frame}

\begin{frame}
    \frametitle{Compilation}

    \begin{gather*}
        \$XY = \lambda x.\; \lambda y.\; - \;y \;x \\ 
        \onslide<2->{ \$XY \;x \;y = - \;y \;x }
    \end{gather*}

    \pause
    \pause

    \begin{center}
        \begin{tabular}{l}
            Rewrite rules: \\
            \(\$XY \;x \;y = - \;y \;x\) \\
            \hline
            Expression to be evaluated: \\
            \(\$XY \;3 \;4\)
        \end{tabular}
    \end{center}

    % note that this is basically a let definitions in this expression block
    % while I don't have a formal proof for you atm, I hope it intuitively makes
    % sense that since if we expect the output of our program to be a
    % first-order term, then we will keep finding supercombinator redexes until
    % we're done evaluating

\end{frame}

\begin{frame}
    \frametitle{Compilation: Example}

    \begin{equation*}
        (\lambda x. \; (\lambda y.\; + \;y \;x) \;x) \;4 
    \end{equation*}

    Notably, no supercombinators right now.

    \pause

    \vspace*{3em}

    \begin{center}
        \begin{tabular}{l}
            \$YX \;y \;x = + \;y \;x \\
            \$X \;x = \$YX \;x \;x \\
            \hline
            \$X \;4
        \end{tabular}
    \end{center}

    \vspace*{3em}

\end{frame}

\begin{frame}
    \frametitle{Recursion, case analysis}

    \begin{center}
        \begin{tabular}{l}
            \$L \;xs = \text{if } (=\; xs \;\text{nil}) \text{then} \;0 \text{else} \;1 + (\$L \;(\text{tail} \;xs)) \\
            \hline
            \$L \; (\text{cons} \;1 \;(\text{cons} \;2 \;(\text{cons} \;3 \;\text{nil}))) 
        \end{tabular}
    \end{center}

    \footnote{Treatment of recursion from Johnsson, Augustsson (1985).}

\end{frame}

\begin{frame}
    \frametitle{Combinator Families}

    Sch\"onfinkel's combinators \(\{S, K, I, B, C\}\) where:
    \begin{align*}
        B \;x \;y \;z &= x \;(y \;z) \\
        C \;x \;y \;z &= x \;z \;y
    \end{align*}

    \pause

    Define a family \(\Phi\) indexed by trees with words as leaves. Here \(w \in \{b, c\}^*\).
    \begin{align*}
        \Phi_\epsilon &= I \\
        \Phi_{b\cdot w} \; x \; x_1 \ldots x_{|w| + 1}&= x \;(\Phi_w \;x_1 \;\ldots \; x_{|w| + 1}) \\
        \Phi_{c\cdot w} \; x \; x_1 \ldots x_{|w| + 1}&= (\Phi_w \;x_1 \;\ldots \; x_{|w| + 1}) \;x \\
        \Phi_{(t_1, t_2)} \; x_1 \ldots x_{|t_1|} \; y_1 \; \ldots \; y_{|t_2|} \; z &= (\Phi_{t_1} \;\bar x \;z) (\Phi_{t_2} \;\bar y \;z)
    \end{align*}

\end{frame}

\begin{frame}
    \frametitle{Combinatory logic in GHC}

    \begin{itemize}[<+->]
        \item Was \emph{maybe} used as the foundation earlier?
        \item Definitely switched by 2004 \\ \fullcite{10.1145/1016848.1016856}
        \item But the core ideas remain
        \item This is why the GHC assembly \emph{looked} familiar before but was still a bit alien
        \item I understand it a bit more now
        \item You can spot blocks of ``rewrites''
        \item Every piece of data is a function, and every function is data
    \end{itemize}

\end{frame}

% extras

\begin{frame}
    \frametitle{On history of combinators and of Sch\"onfinkel}

    Stephen Wolfram (2020), "Combinators and the Story of Computation," Stephen
    Wolfram Writings.
    \url{https://writings.stephenwolfram.com/2020/12/combinators-and-the-story-of-computation}.

    Stephen Wolfram (2020), "Where Did Combinators Come From? Hunting the Story
    of Moses Schönfinkel," Stephen Wolfram Writings.
    \url{https://writings.stephenwolfram.com/2020/12/where-did-combinators-come-from-hunting-the-story-of-moses-schonfinkel}.

    Stephen Wolfram (2021), "A Little Closer to Finding What Became of Moses
    Schönfinkel, Inventor of Combinators," Stephen Wolfram Writings.
    \url{https://writings.stephenwolfram.com/2021/03/a-little-closer-to-finding-what-became-of-moses-schonfinkel-inventor-of-combinators.}

\end{frame}

\begin{frame}
    \frametitle{In interpretations of intuitionistic logic}
    
    \fullcite{nlab:combinatorylogic}
    
    \fullcite{nlab:partialcombinatoryalgebra}

    \fullcite{nlab:realizabilitytopos}

\end{frame}

\begin{frame}
    \frametitle{On Combinator Compilation}

    \fullcite{peytonjones1987the}

    \fullcite{turner1979new}

    \fullcite{Broda_Damas_1997}

\end{frame}

\begin{frame}
    \frametitle{Implementation details}

    SASL:
    
    \fullcite{turner1979new}

    Eventually became the G-Machine:

    \fullcite{10.1145/99370.99386}

    \fullcite{jones1992implementing}

\end{frame}

\begin{frame}
    \frametitle{On comparison of implementation models}

    \fullcite{10.1145/1016848.1016856}

    \fullcite{10.1145/276393.276397}

\end{frame}

\begin{frame}

    \begin{center}
        \huge Thanks! Questions?
    \end{center}

    \vspace*{2em}

    A summary:
    \begin{itemize}
        \item SKI combinators
        \item Supercombinator compilation
        \begin{itemize}
            \item Lambda-lifting
            \item Recursion
            \item Combinator families
        \end{itemize}
        \item A look at remnants in GHC
    \end{itemize}

\end{frame}

\printbibliography

\end{document}
